\documentclass[a4paper]{article}
\usepackage[utf8]{inputenc}
\usepackage[spanish, es-tabla]{babel}

\usepackage{amsmath}
\usepackage{amsfonts}
\usepackage{amssymb}

\usepackage{float}
\usepackage{graphicx}
\graphicspath{ {./Imagenes/} }

\usepackage[american voltages,american currents]{circuitikz}

\usepackage{fancyhdr}

\usepackage{units} 

\pagestyle{fancy}
\fancyhf{}
%\lhead{23.09 Física Electrónica}
\rhead{Bertachini, Lambertucci, Londero, Mechoulam, Musich}
\rfoot{Página \thepage}


\begin{document}

%%%%%%%%%%%%%%%%%%%%%%%%%%%%%%%%%%%%%%%%%%%%%%%%%%%%%%%%%%%%%%%%%%%%%%%%% 
%								CARATULA								%
%%%%%%%%%%%%%%%%%%%%%%%%%%%%%%%%%%%%%%%%%%%%%%%%%%%%%%%%%%%%%%%%%%%%%%%%% 

\begin{titlepage}
\newcommand{\HRule}{\rule{\linewidth}{0.5mm}}
\center
\mbox{\textsc{\LARGE \bfseries {Instituto Tecnológico de Buenos Aires}}}\\[1.5cm]
\textsc{\Large 22.02 Electrotecnia I}\\[0.5cm]


\HRule \\[0.6cm]
{ \Huge \bfseries Trabajo práctico Física Electrónica\\[0.4cm] 
\HRule \\[1.5cm]


{\large

\emph{Grupo 5}\\
\vspace{3px}

\begin{tabular}{lr} 	
\textsc{Mechoulam}, Alan  &  58438\\
\textsc{Lambertucci}, Guido Enrique  & 58009 \\
\textsc{Bertachini},Germán  & 61337 \\
\textsc{Musich}, Francisco  & 57521 \\
\textsc{Londero Bonaparte}, Tomás Guillermo  & 58150 \\
\end{tabular}

\vspace{20px}

\emph{Profesores}\\
\vspace{3px}
\textsc{Gardella}, Pablo\\ 	
\textsc{Baez}, Eduardo\\ 	
\textsc{???}, Analia\\

\vspace{100px}

\begin{tabular}{ll}

Presentado: & 24/05/19\\

\end{tabular}

}

\vfill

\end{titlepage}


%%%%%%%%%%%%%%%%%%%%%%%%%%%%%%%%%%%%%%%%%%%%%%%%%%%%%%%%%%%%%%%%%%%%%%%%% 
%								INFORME									%
%%%%%%%%%%%%%%%%%%%%%%%%%%%%%%%%%%%%%%%%%%%%%%%%%%%%%%%%%%%%%%%%%%%%%%%%%

\section*{Introducción}

En el trabajo presente se llevó adelante el estudio de distintos tipos de diodos y circuitos, con el objetivo de llevar a la práctica la teoría estudiada en clases.

\section*{Desarrollo de la experiencia}

Primero se analizaron tres diodos distintos: rectificador 1N4148, zener y LED rojo, conectados de la forma presentada en el circuito (\ref{circ:1}).

\begin{figure}[H]
\begin{center}
\begin{circuitikz}
\draw
	(4,0)	to (0,0)
	(0,1.5)	to [sV,v_=$V$]	(0,0)
	(0,1.5)	to [R=$ 470 \ \Omega $]	(4,1.5)
	(4,0)	to [Do, l_=Diodo analizado]	(4,1.5)
;\end{circuitikz}
\end{center}
\caption{Circuito utilizado para medir los diodos.}
\label{circ:1}
\end{figure}

Utilizando multimetros se observó el comportamiento de estos para ciertas tensiones y posteriormente se graficaron las curvas correspondiente a cada uno.

\begin{figure}[H]
	\centering
	\includegraphics[width=0.6\textwidth]{CurvaDiodoRectificador}
	\caption{Corriente en función de la tensión del diodo rectificador 1N4148.}
	\label{fig:diodorect}
\end{figure}

\begin{figure}[H]
	\centering
	\includegraphics[width=0.6\textwidth]{CurvaZenerEntera}
	\caption{Corriente en función de la tensión del diodo zener.}
	\label{fig:diodozen}
\end{figure}

\begin{figure}[H]
	\centering
	\includegraphics[width=0.6\textwidth]{CurvaDiodosLed}
	\caption{Corriente en función de la tensión del diodo LED.}
	\label{fig:diodoled}
\end{figure}

Luego se simuló el siguiente circuito

\begin{figure}[H]
\begin{center}
\begin{circuitikz}
\draw
	(0,1.5)	to	[sV,v_=$AC$]	(0,-1)
	(0,1.5)	to	[R=$R_G$]	(1.5,1.5)
	(1.5,1.5)	to	[C, l=$C_{in}$]	(3,1.5)
	(0,-1)	to	(4.5,-1)
	(3,-1)	to	[R, l=$R_2$]	(3,1.5)
	(3,1.5)	to	[R, l_=$R_1$]	(3,4)
	(3,4)	to	[V,v_=$VCC$]	(1,4)	node[sground]{}	
	(3,4)	to	(4.5,4)
	(3,1.5)	to	(4,1.5)
	
	(4.5,1.5) node[npn,name=npn]{}
%	(3,1.5)	to	(4.5,1.5) node[npn,name=npn]{}

	(4.5,4)	to	[R, l=$R_C$]	(4.5,2)
%	(4.5,4)	to	[R, l=$R_C$]	(npn)

	(4.5,0.75)	to	[R, l_=$R_E$]	(4.5,-1)	

	(4.5,-1)	to	(7,-1) node[sground]{}	
	(4.5,0.75)	to	(6,0.75)
	(6,0.75) to [C, l_=$C_{E}$]	(6,-1)
	
	(7,-1)	to	[R, l_=$R_L$]	(7,2)
	(7,2)	to	[C, l_=$C_{out}$]	(4.5,2)
	
;\end{circuitikz}
\caption{Circuito analizado en la segunda instancia.}
\end{center}
\label{circ:2}
\end{figure}

Analizando la función transferencia de tensión de este, se observa que, para una $V_{in}$ arbitraria
\begin{equation}
H(s) = \frac{V_{out}}{V_{in}} = \frac{5.22 \cdot 10^{-16} \ V}{5 \ V} \approx 0
\end{equation}

Finalmente se analizó la respuesta en frecuencia del circuito (\ref{circ:3}).

\begin{figure}[H]
\begin{center}
\begin{circuitikz}
\draw
	(0,0)	to (4,0)
	(0,1.5)	to [sV,v_=$AC$]	(0,0)
	(0,2.5)	to [V,v_=$DC$]	(0,1.5)
	(0,2.5)	to [R=$ R_G $]	(2,2.5)
	(2,2.5)	to [R=$ R $]	(4,2.5)
	(4,0)	to [Do, l_=D1N4007]	(4,2.5)
;\end{circuitikz}
\end{center}
\caption{Circuito utilizado para medir los diodos.}
\label{circ:3}
\end{figure}

\begin{figure}[H]
	\centering
	\includegraphics[width=0.6\textwidth]{RtaF3_2}					%DETERMINAR SI USAR RtaF3_1 O RtaF3_2
	\caption{Respuesta en frecuencia del diodo D1N4007.}
	\label{fig:rtaf}
\end{figure}

\section*{Conclusión}

Los resultados obtenidos al estudiar los tres primeros diodos se corresponden con los resultados esperados. Se observan pequeñas incongruencias en los gráficos, como por ejemplo en el gráfico (\ref{fig:diodorect}), que son atribuidos a cambios de escala de los instrumentos durante el análisis.

Luego se comprobó que, la función transferencia del segundo análisis, puede ser considerada nula, sin importar la tensión de entrada ...

Por último, la respuesta en frecuencia mostró ...


\end{document}

